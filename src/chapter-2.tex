In this chapter, we talk about the state of the art in computer vision and how it has been used for remote sensing problems. We also give a brief account of natural disaster assessment, and how are these machine techniques applied in this sense. We use the Chiapas earthquake that happened on Septmber 7, 2017 as a study case because of the data that was publicly made available by the CENAPRED. In this chapter the mathematical details of the CNN are exposed. How does backpropagation works, and differences between multiclass and multilabel classification.\\



A wonderful introduction to neural networks in the context of remote sensing can be found in \cite{canty2014image}. It reviews all the necesary mathematical tooling to deal with the processing image algorithms, and implements the algorithms in Python. 

In the context of natural disaste rs other options have been considered. For example Kryvasgeteue et. al. use the twitter activty during  Hurricane Sandy to build a model. They found that social media activity is related with the proximity of the region in which the Hurracane hits\cite{Kryvasheyeue1500779}. Nevertheless, the mention that the nature of the data available both in quantity and quality is rather unususal in part because of the severity and magnitude of damage that the Hurricane Sandy brought about.\\

The idea of using features crafted by a neural network has been widely explored. Jeff Donahue et. al. developed a framework which they called Deep Convolutional Activation Features (DeCAF) \cite{DBLP:journals/corr/DonahueJVHZTD13}. They feed traditional methods with the features extracted from the CNN obtaining accuracies compared to the state of the art at the time.\\

They took the activations from the $n^{th}$ hidden layer, the layer previous to the fully connected one, and use them to train two classifiers: a logistic regression, and a support vector machine, in four different environments: object recognition, domain adaptation, subcategory recognition and scene recognition. Object recognition tested the ability of the classifier to correctly assign a class to a given image. They showed that the depth of the layer dramatically improves the performance of the features. Domain adaptaion consist of testing the robustness of the classifier when the source of the image varies, in this case they use a testing set consisting of images of office objects taken from Amazon, a webcam and a Dslr camera. The classifiers where able to cluster the objects across domains regardless of the origin of the images. In subcategory recognition, the task at hand is to differenciate individual categories inside a super-category, for example discriminate among two types of birds using a training set consisting only of images of birds. They found that without further fine tuning, the features extracted from the neural network achieved accuracies far superior to the ones available in the literature at the time. Finally, they tested the extracted features at the task of semantical classification. Instead of requiring the classifier to give a category of the object in the picture, the classifier must semantically classify the scene instead. This is considered a very difficult task, and the previous approach at the moment was to use hand-engineered features and a multi-kernel learning baseline. While the accuracies reported by this task where quite low, they improved the existing benchmark which provides strong evidence that the features extracted from the deeper layers of a CNN excel at extracting information from the images even though the network was not design for any of these tasks.\\


Additionaly they developed an open source framework that later matured into Caffe \cite{jia2014caffe}, one of the most complete frameworks for creating CNN models\\

This is another attempt that adds to the evidence that features engineered by the Neural Network work pretty good off the shelf. \cite{DBLP:journals/corr/RazavianASC14}\\

Transfer learning is explored by Yosinski \textit{et al.} \cite{DBLP:journals/corr/YosinskiCBL14}. They propose to use an already trained architecture in new tasks by replacing diferent layers and retraining.\\

In \cite{DBLP:journals/corr/LongSD14} they use the features extracted from the CNN to segment an image.\\

The possibility of having a single model that can perfom correctly in many different tasks is explored in \cite{DBLP:journals/corr/KaiserGSVPJU17}\\

A survey on disctint aspects of how transfer learning is used can be found in \cite{5288526}.\\

The use of active learning together with semisupervised learning tenchniques is explored in \cite{7956153}.\\

Need to take a look into \cite{DBLP:journals/corr/ChenPKMY14}\\

Everardo suggested to look into the U net \cite{DBLP:journals/corr/RonnebergerFB15}.


And the books: \cite{canty2014image}, \cite{richards2013remote}, \cite{tso2009classification} ,\cite{hastie01statisticallearning}




\section{Computer vision}

Le Cun \textit{et al.} \cite{119325} propose to use an architecture of a multilayer neural network that was able to learn directly from the data with no prior feature extraction. In contrast to the usual path that was used in the context of pattern classification, they created an architecture that was able to automatically extract the features directly from the date without prior manipulation. Instead of using a fully connected network, they proposed a locally connected net. It was capable of extracting local features and passed them down to the subsequent layers in what they called a \textit{feature map}. Each unit took the information of a $5\times 5$ neighborhood of the pixel in the previous layer. The last layer of the architecture consisted of ten units that represented each of the possible digits. This architecture was trained using backpropagation are now known as Convolutional Neural Networks (CNNs). The big leap forward of their result was that their architecture needed very little information about the task it was performing, they were able to extend the use of their method to other symbols, however, they state that the method was not able to be applied to very complex objects.\\

Before CNNs became widely used in computer vision tasks, the use of neural networks to automatically detect roads was already being explored \cite{Mnih:2010:LDR:1888212.1888230}. They proposed a simple architecture with a single hidden layer. They use the now standard procedure of cropping small patches from the complete image and worked on the RGB color space. Aerial imagery in addition to vector information on roads is used as training data. Instead of manually tag the images they proposed a model that asumes certain amount of thickness in the roads, which are tipicaly with no dimension. To reduce the input dimensionality, they applied Principal Component Analysis keeping the most informative principal components. As a method of post-processing, they use a CNN to reduce the noise in the images. Effectively getting rid of false positives and false negatives by using context information. Of the important lessons learned by this experiment is the value of the random rotations in the training data. It is common that road networks in cities form grids, they realised that a model trained with information from a particular city will perform poorly if data from a new city is shown to it. They found that this can be relived if random orientations are applied to data as when we are dealing with birdeye sight, there is no correct orientation for images. They also found that adding pre-procesing such as edge detection techniques showed no improvement to the performance of their pipeline. This was attributed to the fact that the neural network crafts features of this nature when learning to perform the task of recognizing roads.

There have been several attempts to use features extracted from a CNN to be used in a different context. Michael Xie \textit{et al.} \cite{DBLP:journals/corr/XieJBLE15} examine this approach by training a CNN on top of the well-known \texttt{VGG F} model. First, they replace the fully connected layers on top of that model with a convolutional layer. Then they re-train the features the model learned from ImageNet with aerial images and nighttime images gather from the National Oceanic and Atmospheric Administration (NOAA). While they get nice accuracy results from using daytime images to predict nighttime light, it was not the purpose of their research. Instead, features crafted by the network are extracted and used to train a model to predict poverty from satellite imagery. In order to do so, they use these features as input for a logistic regression classifier. To compare their model, they train four other models extracting features from a survey, features from ImageNet itself, features from the nighttime light intensities and features from ImageNet and nighttime light intensities at the same time. The transfer model outperforms every model except for the one based on survey data. This strongly suggests that the transfer learning technique is actually extracting complex information from the aerial scenes. They mention that this approach can be useful when conducting surveys is prohibitively expensive.

With the tremendous advances that computer power has suffered in the late years, this has been proven to be incorrect. In 2009 a big image database was gathered and published \cite{Deng09imagenet:a}. Ever since this database became the defacto dataset to test classification methods. A few years later, in 2012 Krizhevsky \textit{et al.} \cite{krizhevsky} proposed the use of CNNs in this daunting task.\\

\section{Remote sensing}

In late years groundbreaking advances in computer vision have led to tremendous advances in other science fields. In particular, we are interested in landcover classification.\\

The use of CNNs in the context of landcover classification was explored by Kussul \textit{et al.} \cite{7891032}. They used an ensemble of CNNs to obtain state of the art results in the classification of different types of crops using multitemporal and multisensor satellite data. They explore 2 approaches, first they use a 1-D CNN to perform the convolutions in the spectral domain by stacking the different bands from the Sentinel-1 A and Landsat-8 scenes. This process outputs a pixel-wise classification, then they perform a traditional 2-D CNN on the scenes. In order not to lose resolution with the 2-D CNN, they use a sliding window approach assigning the class to the center pixel of the sliding window. Finally, they ensemble both opinions and filter the result to improve the quality of the map.

The usual approach with landcover classification is the use of classical classification methods such as support vector machines (SVM) and random forests (RF). In order to improve the performance, features must be handcrafted from the original bands. In \cite{7858676}, Grant \textit{et al.} explore the use of Transfer Learning and Data Augmentation in the context of remote sensing images. By exploring well-known high-resolution datasets, they obtain state of the art results.\\


Segnet papers: \cite{DBLP:journals/corr/BadrinarayananK15} in \cite{DBLP:journals/corr/KendallBC15} they enhace their approach by extending their own architecture to include a Bayesian approach. The idea is to add a model of uncertainty to the CNN and use this information to get more acurrate guesses on each of the pixels. They report that this feature adds some improvement in the level of accuracy for many types of architectures not only their own SegNet.





\subsection{Data augmentation}

Data augmentation is a technique used to artificially increment the size of the training dataset by applying an affine transformation to the images. It is often used when tagged data is scarce and difficult to obtain. The usual transformations include rotations and reflections. When using this technique we should be careful about the orientation of the objects, for example, a building upside down makes no sense, so there is no use to make the network learn features on objects that it won't see in the wild. Fortunately, aerial imagery doesn't present this problem. There is no particular orientation that can be considered correct when the pictures are taken from above. This means that we can dramatically boost the size of our dataset.\\

The reasoning behind this idea is that when we see a picture, our brain automatically orients it into its correct position. By showing the network with different positions and orientations of an object we enrich its knowledge about it.\\

We can think of the neural network as a newborn kid, in the beginning, they experience its environment for the first time.\\



\section{Damage Assessment}



\section{TensorFlow}

TensorFlow is a machine learning system developed by the Google Brain Team to supersede its first-generation system, DistBelief. It was built on top of the lessons learned during DistBelief development. One of the fundamental concepts that lead the team to create a new system from scratch was the need for flexibility. TensorFlow was thought as a way to expressing machine learning algorithms using a common interface with implementations targeting a wide range of devices. Complex models that where first implemented in DistBelief such as Inception got a performance boost of a factor of x6 \cite{tensorflow2015-whitepaper} when it was ported to the new system. TensorFlow was opened sourced on November 9, 2015 under the Apache 2.0 license.\\ 

In this section, we will untangle some of the details of how TensorFlow and its graph model work \cite{DBLP:journals/corr/AbadiBCCDDDGIIK16}.\\

It uses an elegant data-flow system in which both the operations and the state of the algorithm are represented as nodes and edges in a directed graph. This lets the system to pre-calculate an optimal sub-graph before starting the calculations.\\

With flexibility in mind, this system lets the user define new operations and register them to use within the framework. Additionally, it was developed to target different platforms, from machine clusters to mobile devices, these implementations are known as \textit{kernels}. Using the same programming model, TensorFlow decides in runtime which pieces to use. This is useful when you take into account the whole development process of a data product and how it evolves. We can think of a common scenario, first, the developer experiments with data in a single computer before deploying the system to train with a larger data set in a cluster of computers, when the model is trained, it can be deployed to an online service which will run on a single computer or it can be implemented to be used in a mobile device for offline use. In each of these steps the underlying environment is completely different, however, TensorFlow adapts automatically to each situation.\\


As a common interchange data format, it uses tensors. With machine learning algorithms, it is often the case to have sparse data, encoding it as dense tensors is a clever way to save space. As we mentioned before, TensorFlow uses a graph to represent both the state and the operations. Nodes represent operations. Edges represent inputs and outputs between these operations. The system takes its name from the tensors flowing through this pipes. Although it is not of particular importance to our experiment, it is worth mentioning that TensorFlow supports algorithms with conditional and iterative control flow, which means that it can be used, without further tuning, to train Recurrent Neural Networks which are very important in fields like speech recognition and language modeling.\\

The system also provides a library that allows symbolic differentiation. As many machine learning techniques rely on Stochastic Gradient Descent to train a set of parameters, this feature makes easier to explore new techniques as the backpropagation code is automatically produced for any combination of operation nodes.\\

TensorFlow was built with huge data-sets in mind. It provides an intern library that allows the distribution of datasets that would be to large to fit in RAM. Instead, data can be sliced, taking advantage of how some algorithms work. Additionally, comunication between nodes use lossy compression, taking advantage of the fact that some of the machine learning algorithms are tolerant to reduced precision arithmetic. It is important to mention that it is possible to extract state and information from any particular node in the graph. This fact is very important to our study because we are interested in the features that the system craft to perfom the given task. We want to teach the system to excel at our task of interest and then use its knowledge to improve another task.\\

Another feature that TensorFlow provides is its ability to prune the execution graph before starting its computations. Usually several sets operations are repeated along the graph, by detecting this, the system can automatically replace all incidences of each repeated sub-graph with a single one thus, saving memory and time. The same case happens with communication nodes. If several nodes from a single device are consuming the same data from another device, the system is prepared to detect this and ensure that the data transfer occur only once.\\

The framework code is deeply optimized. Implementations of the same interface target the different dispositives that the code can run in. It is built upon known mature frameworks such as cuDNN, a library for deep neural networks that targets NVIDIA GPUs, and Eigen, a C++ library for linear algebra, which was extended to offer tensor arithmetic support. On top of this infrastructure, TensorFlow offers a Python client which is very convenient for fast development.\\

An interesting tool that is packed with the framework is TensorBoard. With the huge complexity that machine learning models offer at this scale, it is important to know what is happenining at any point of the training process. TensorBoard offers a glimpse of how does the architecture for a particualar computation graph looks like. This will become handy later when describing our model.\\

There exist several systems that offer similar features to TensorFlow. Theano, Torch, Caffe, Keras to name a few. The purpose of this work is by no means to study the advantages or disadvantages of these systems nor to create a benchmark on their perfomance. As the pipeline was already written in Python, we choose TensorFlow to minimize the gluing code. Among the examples that come with the system there is one end to end example of how to transfer learining from one trained net to another task.\\











