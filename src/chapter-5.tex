


This was an important experiment because it let us describe the use of novel techniques in the process of image classification.

\section{Conclusion}

Our efforts showed that it is possible to deliver a preliminary product with little training effort. This product might be used as a baseline in the event of a disaster such as an earthquake to give an idea of where to start looking for damaged buildings so resources can be allocated in an efficient manner. 

\section{Drawbacks}

We noticed that, even when the classifier performs well in environments different to the one that it was trained with, it learns to classify dribble, not exactly damaged buildings. We noticed some examples in which the classifier correctly finds scenes with presence of dribble, however, when we inspected the place using Google Street View, we noticed that there was no building in that place. This can can think of two possible scenarios in which this is possible; there was never a building in that place an it was used as a disposal for dribble from other places, or a house was built after the picture in Google Street View was taken. Bothe cases show inherent limitations of the methodology that we are proposing, we can only automate this kind of process to a certain extent.

\section{Future work}

An idea that was beyond the scope of this reasearch was to incorporate a technique known as active learning. In this scheme, the algorithm keeps improving as it receives feedback from the experts. In this fashion, when the model makes mistakes, this incorrectly labeled scenes can be relabeled and feed back to the system in order to create a new model and keep improving its performance. Although there is a limit to the extent in which the algorithm can perform, this might aid in some obvious cases that might not be present in the original training data.\\

Another aspect that can be explored is the application of the same analysis framework in other type of studies. CNN have proven to be very perfomant in tasks that other classic computer vision algorithms fail to perform correctly. In the National Commission for the Knowledge and Use of Biodiversity we use landcover maps to analyze and assess the evolution of the environment through time. We make this possible by leveraging classic classification algorithms and a large amount of computing power. While our efforts have been quite productive, these algorithms have certain limits, they rely on the use of the light spectrum. As a consequence, any two categories with similar spectrum footprint will, in all likelihood, confuse the classifier. Curiously enough, humans have little problem distinguish between some of these pairs of categories. For example, crops and grasslands might seem identical to a supervised classifier, but we can certainly tell the difference from one another. This is caused because our brains are not seeing particular pixels and trying to classify them one by one. Instead, our brains look at the whole picture, we focus on zones of the image an all of the information included in them, in other words, we care about context. Convolutional Neural Networks take this into account. Each neuron of the network cares only about a certain zone of the image when information flows through the layers of the network, there are certain neurons that activate upon certain stimuli. Taking context into account lets the network to recognize certain features that would be invisible to a classic classifier, for example: shapes and geometries.\\

When we ask ourselves why it is so easy to differentiate crops from grasslands geometry comes as a natural answer. Crops have very particular shapes.\\

The final objective is to build a comprehensive biodiversity monitoring system. It can be though as two independent efforts. One of these attemps is the Monitoring Activity Data for the Mexican REDD+ program (MADMex) \cite{rs6053923} which pretends to monitor the behavior of forest and vegetation across the country by processing satelitte imagery. Another effort is the Mexican National Biodiversity and Ecosystem Degradation Monitoring System (SNMB) \cite{GARCIAALANIZ201762} which gathers information about species in the different ecosystems that exist in Mexico.\\

