


Even though deep learning has allowed us to do a giant leap forward on the performance of several tasks, and contrary to what some researchers claim \cite{DBLP:journals/corr/KaiserGSVPJU17}, there exists no universal model. A good model would be able to be performant in very different settings, but it won't be able to handle every situation. We need our model to be robust, but we can not expect it to be perfect. Being aware of the limitations intrinsic to the model is the best way to alleviate possible mistakes. Although it sometimes seems the case, deep learning is not a magic box, it learns from the data that we supply. If the model is biased, in all likelihood, it means that our training set is biased.\\

Ortorectified mosaics where built by CENAPRED using the very same images taht we used to train our models. This images come in a different format than the images taken by the drones. Additional to the optical information these tif files contain geographical location and can be used to assing a point in space to each pixel in the image. This means that we can not only locate a damaged building in an image, but to link this information with a geographical location in a given projection.\\

The images where divided in a regular grid of 299 pixel tiles with 90 pixel ovelaps. These overlaps are later postprocessed to eliminate the posibility of counting the same building twice. Each tile is exposed to the model which predicts a class on it using the previously selected threshold. When the model test a tile positive, the box is saved for postprocesing in which a technique known as non max suppression is used to eliminate boxes that represent the same object. This technique is borrowed from facial recognition algorithms. Once we have the final boxes, the center pixel of each box is transformed to world coordinates. Additionaly, this coordinates are used to query Google Maps API to obtain a human readable address for each point.\\

A shape file is produced which contains the results for each town given the output of the algorithm. This shape can be overlayed on top of the raster file using a Geographic Information System software such as QGIS. Additionaly the results are also exposed via the REST interface so they can be visualised in the web application.\\



\begin{center}
  \begin{tabular}{|c|c|c|c|c|c|}
    \hline
    town                 & positives & width & height & time (seconds) & overlap\\ \hline
    Santa Maria Xadani   &51         & 25598 & 30144  & 4420           & 0.1 \\ \hline
    Juchitan de Zaragoza &302        & 42375 & 28831  & 6375           & 0.1 \\ \hline
    Union Hidalgo        &25         & 19945 & 28795  & 3938           & 0.1\\
    \hline
  \end{tabular}
\end{center}


