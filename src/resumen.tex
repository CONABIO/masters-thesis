Los terremotos son un fen\'omeno impredecible y de inevitable naturaleza cuyo nivel de destrucci\'on puede llegar a catastr\'ofico. Debido a esto, buscar alternativas que ofrezcan una respuesta efectiva y r\'apida es una prioridad tanto para la \'optima canalizaci\'on de recursos como en el proceso de mitigaci\'on de las secuelas.\\

Tras el sismo del 19 de septiembre de 2017 en la Ciudad de M\'exico la colaboraci\'on masiva demostr\'o ser un recurso fundamental para afrontar el desastre. El uso de las redes sociales permiti\'o la sinergia entre labores de rescate, log\'isticas y las acciones de la sociedad civil. Sin embargo, ?`qu\'e ocurre cuando no existen las condiciones ni la infraestructura tecnol\'ogica de las grandes urbes? Este trabajo propone un modelo aplicable a estos casos.\\

Durante los d\'ias posteriores al sismo del 7 de septiembre de 2017, con epicentro en el Golfo de Tehuantepec, drones del Centro Nacional para la Prevenci\'on de Desastres (CENAPRED) inspeccionaron las zonas afectadas sobre tres pueblos en el estado de Oaxaca. Las im\'agenes obtenidas fueron liberadas y sirven como objeto de estudio para esta investigaci\'on.\\

Partiendo de una red neuronal convolucional entrenada con un banco de im\'agenes dise\~nado para la investigaci\'on en el campo de reconocimiento de objetos, se ajust\'o un modelo que detecta edificios da\~nados en las fotos a\'ereas. Con ayuda de t\'ecnicas de georectificaci\'on, el modelo predictivo se us\'o para generar un mapa de da\~nos potenciales. Los resultados sugieren que este proceso podr\'ia servir para generar herramientas que ayuden a la correcta asignaci\'on de los recusos para minimizar costos y facilitar su operaci\'on.\\